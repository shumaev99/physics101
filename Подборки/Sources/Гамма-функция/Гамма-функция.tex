\documentclass[14pt]{article}

\usepackage[russian]{babel}
\usepackage[utf8]{inputenc}
\usepackage{amsmath,amssymb}
\usepackage{parskip}
\usepackage{textcomp}
\usepackage{gensymb}
\usepackage{setspace}


\oddsidemargin=0 cm
\evensidemargin=0 cm
\textwidth=170 mm
\textheight=230 mm
\topmargin=0 cm
\voffset= -2cm
\pagenumbering{false}
\newlength{\varheight}
\setlength{\varheight}{3.1cm}
\setlength{\parindent}{0cm}
\spacing{1.1}
\parskip=2mm
\clubpenalty=10000
\widowpenalty=10000

\begin{document}

\begin{center}
\Large{\textbf{Гамма -- функция}}

\textbf{01.03.2018}

\vspace{5mm}
\end{center}

Обозначим

$$
  \Gamma(s)=\int\limits_0^\infty x^{s-1}e^{-x}dx.
$$

Функцию $\Gamma(s)$ называют гамма-функцией аргумента $s$.

1. При помощи интегрирования по частям найдите $\Gamma(s)$ при $s=0,1,2,3$. Какую закономерность вы наблюдаете?

2. Найдите рекуррентную формулу, выражающую $\Gamma(s+1)$ через $\Gamma(s)$.

3. Решите получившееся рекуррентное соотношение при целых $s$, т.е. выразите $\Gamma(s)$ через $s$.

4. Найдите $\Gamma(3/2)$ при помощи замены переменных.

5. Чему равно $\Gamma(s+1/2)$ при целых $s$?

\end{document} 