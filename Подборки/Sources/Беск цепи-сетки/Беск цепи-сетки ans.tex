\documentclass[14pt]{article}

\usepackage[russian]{babel}
\usepackage[utf8]{inputenc}
\usepackage{amsmath,amssymb}
\usepackage{parskip}
\usepackage{caption}
\usepackage{textcomp}
\usepackage{gensymb}
\usepackage[dvips]{graphicx}
\usepackage{wrapfig}
\usepackage{color}
\usepackage{setspace}
%\usepackage{hyperref}
\usepackage{epstopdf}

\oddsidemargin=0 cm
\evensidemargin=0 cm
\textwidth=170 mm
\textheight=230 mm
\topmargin=0 cm
\voffset= -2cm
\pagenumbering{false}
\newlength{\varheight}
\setlength{\varheight}{3.1cm}
\setlength{\parindent}{0cm}
\spacing{1.1}
\parskip=2mm
\clubpenalty=10000
\widowpenalty=10000
\captionsetup[figure]{labelformat=empty}

\begin{document}

\begin{center}
\Large{\textbf{Бесконечные цепи и сетки --- ответы}}
\end{center}

\vspace{5mm}

1. а) $r=R\cdot\dfrac{1+\sqrt5}{2}$,\hspace{5cm} б) $r=\dfrac{R_1}{2}\left(1+\sqrt{1+\dfrac{4R_2}{R_1}}\right)$.

2. $R_{AB}=R\cdot\dfrac{1+\sqrt{21}}{5+\sqrt{21}}$.

3. $r_1=R\left(-1+\dfrac{\sqrt{17}}{2}\right)$, $r_2=R\left(-\dfrac{3}{2}+\dfrac{\sqrt{17}}{2}\right)$.

4. 

5. $E=\varepsilon\cdot\dfrac{3+\sqrt{5}}{2}$, $R=r\cdot\dfrac{1+\sqrt5}{2}$.

6. $R=\dfrac{2\rho a}{\sqrt7}$.

7$^*$. а) $\varphi=2\arcsin\dfrac{\omega}{2\omega_0}$ при $\omega<2\omega_0$.
\hspace{3cm}
б) $v=\dfrac{\omega l}{\varphi}=\dfrac{\omega l}{2\arcsin\dfrac{\omega}{2\omega_0}}$.

в) При $\omega\ll \omega_0$ скорость $v_0\approx \omega_0 l$.

8. а) $R=r/2$,
\hspace{2cm}
б) $R=r/3$,
\hspace{2cm}
в) $R=r/3$,
\hspace{2cm}
г) $R=2r/3$.

9. а) $R=3r/8$,
\hspace{5cm}
б) $R=r$.

10. $R_3=2R_2-R_1=2R_2-r/2$.

Источники:

2. Винницкий турнир чемпионов 2016, теор старшей группы, задача 2

3. Допы по электрике (московские олимпиады), задача 3.52

4. Отбор на всеукр Харьковской области 2014, тур Майзелиса, 9-1

5. Областная олимпиада Харьковской области 2015, 11-5

6. Областная олимпиада Харьковской области 2011, 9-5

7. IPhO 1987.3

\end{document} 