\documentclass[12pt,a4paper,pdflatex]{disser}
%\usepackage[russian]{babel}
%\usepackage[utf-8]{inputenc}
%\usepackage{amsmath,amssymb}
%\usepackage{longtable}
\usepackage{parskip}
\usepackage{caption}
\usepackage{textcomp}
\usepackage{gensymb}
\usepackage[dvips]{graphicx}
%\usepackage{wrapfig}
%\usepackage{amssymb}*
%\usepackage{color}
%\usepackage{ulem}
\usepackage{setspace}
\usepackage{hyperref}

\oddsidemargin=0 cm
\evensidemargin=0 cm
\textwidth=170 mm
\textheight=260 mm
\topmargin=0 cm
\voffset= -2cm
\pagenumbering{true}
%\newlength{\varheight}
%\setlength{\varheight}{3.1cm}
\setlength{\parindent}{0cm}
%\newcommand{\taskname}[name]{\begin{center} \bf{\Large{name}} \end{center}}
\spacing{1.1}
\parskip=2mm
%\captionsetup[figure]{labelformat=empty}
\clubpenalty=10000
\widowpenalty=10000

\begin{document}

1. Due to Newton law for rotation about the axis of the pendulum
$$
  ml^2 \ddot{\theta}=m\omega^2 l^2\sin\theta \cos\theta-mgl\sin\theta,
$$
which implies
$$
  \ddot{\theta}=\omega^2 \sin\theta \cos\theta-\frac{g}{l}\sin\theta.
$$
For a small perturbation $\delta\theta$
$$
  \delta\ddot{\theta}=\left(\omega^2-\frac{g}{l}\right)\delta\theta.
$$
So, the equilibrium at $\theta=0$ becomes unstable if $\ddot{\theta}>0$, or $\omega>\sqrt{g/l}$. The new equilibrium corresponds to
$$
  \cos\theta=\frac{g}{\omega^2 l}.
$$
For $\omega\le\sqrt{g/l}$ and $\theta=0$ the frequency of oscillations
$$
  \Omega=\sqrt{\frac{g}{l}-\omega^2}.
$$
For $\omega>\sqrt{g/l}$
$$
  \delta\ddot{\theta}=\delta\theta\left(\omega^2 \cos 2\theta-\frac{g}{l}\cos\theta\right),
$$
which implies
$$
  \Omega=\sqrt{-\left(\omega^2 \cos 2\theta-\frac{g}{l}\cos\theta\right)}=\sqrt{\omega^2-\frac{g^2}{\omega^2 l^2}}.
$$

2. The described system has a zero-frequency mode corresponding to one-directional rotation. As the system has 3 bonds, other 2 modes exist. Let $x_i$ be the deviations of the balls. Then
\begin{gather*}
  \ddot{x}_1=\omega_0^2 \left(x_2+x_3-2x_1\right),\\
  \ddot{x}_2=\omega_0^2 \left(x_1+x_3-2x_2\right),\\
  \ddot{x}_3=\omega_0^2 \left(x_1+x_2-2x_3\right),
\end{gather*}
where $\omega_0=\sqrt{k/m}$. Let's look for a solution of the form $x_1=A_1 \sin\omega t$, $x_2=A_2\sin\omega t$, $x_3=A_3\sin\omega t$. Substituting $x_i$ into the motion equations we get
\begin{gather*}
  -\omega^2 A_1=\omega_0^2 \left(A_2+A_3-2A_1\right),\\
  -\omega^2 A_2=\omega_0^2 \left(A_1+A_3-2A_2\right),\\
  -\omega^2 A_3=\omega_0^2 \left(A_1+A_2-2A_3\right).
\end{gather*}
This is a system of linear equations with variable determinants $\Delta_x=\Delta_y=\Delta_z=0$. According to the general theory of linear algebra, this system has a non-zero solution iff the main determinant $\Delta=0$. Thus we get
$$
  \begin{vmatrix}
  2\omega_0^2-\omega^2 & -\omega_0^2 & -\omega_0^2\\
  -\omega_0^2 & 2\omega_0^2-\omega^2 & -\omega_0^2\\
  -\omega_0^2 & -\omega_0^2 & 2\omega_0^2-\omega^2
  \end{vmatrix}=0\Leftrightarrow
  \begin{vmatrix}
  3\omega_0^2-\omega^2 & 0 & 0\\
  0 & 3\omega_0^2-\omega^2 & 0\\
  0 & 0 & 3\omega_0^2-\omega^2
  \end{vmatrix}=0\Leftrightarrow \omega^2=3\omega_0^2.
$$
This equation has a duplicate solution $\omega_{1,2}=\omega_0 \sqrt{3}$. So, the system has an extra mode with frequency $\omega=\omega_0 \sqrt{3}$. It may be proved that the amplitudes $A_i$ of the oscillations may be arbitrary satisfying the relation
$$
  A_1+A_2+A_3=0,
$$
i.e. no motion of the ``center of mass''. The arbitrarity of the amplitudes (or, in fact, of their ratios) is in fact an extra degree of freedom which corresponds to the multiplicity 2 of the root. This may be understood as two modes of oscillation degenerated into one, but with arbitrary amplitudes.

3. Let $\overrightarrow{p}_\gamma$ be the momentum of the incident photon, $\overrightarrow{p}_p$ and $\overrightarrow{p}_\pi$ be the momentum of the proton and the pion respectively after the reaction. Then due to conservation laws
\begin{gather*}
  \overrightarrow{p}_\gamma=\overrightarrow{p}_p+\overrightarrow{p}_\pi,\\
  p_\gamma c+m_p c^2=\sqrt{m_p^2 c^4+p_p^2 c^2}+\sqrt{m_\pi^2 c^4+p_\pi^2 c^2}.
\end{gather*}
It may be proved that in case of constant momentum the energy of the products is minimal when $p_\pi=0$. Then the equations may be simplified:
$$
  p_\gamma c+m_p c^2=\sqrt{m_p^2 c^4+p_\gamma^2 c^2}+m_\pi c^2,
$$
which implies
$$
  p_\gamma=m_\pi c \frac{2m_p-m_\pi}{2\left(m_p-m_\pi\right)}.
$$
Then the threshold energy of the photon is
$$
  \varepsilon_\text{min}=p_\gamma c=m_\pi c^2 \frac{2m_p-m_\pi}{2\left(m_p-m_\pi\right)}=146.3 \text{ MeV}.
$$

4. Let $Ox$ axis be parallel to the plane, $Oy$ axis be perpendicular to it, both axes are in the plane of the dipole. The conducting plane acts as as a mirror, creating an image dipole. As the electric field exists only in half of the space, the energy is half the energy of the dipole-dipole interaction. The original dipole is $\overrightarrow{p}=(p\sin\theta,p\cos\theta)$, the image is $\overrightarrow{p_1}=(-p\sin\theta,p\cos\theta)$. The first dipole creates a field
$$
  \overrightarrow{E}=\frac{4\pi\varepsilon_0}{r^3}\left(-\overrightarrow{p}+\frac{(\overrightarrow{p},\overrightarrow{r})\overrightarrow{r}}{r^2}\right),
$$
where $\overrightarrow{r}$ is the radius-vector measured from the center of the dipole, $\overrightarrow{r}=(0,-2h)$ for the location of the second dipole. Then the energy of their interaction is
$$
  W=-(\overrightarrow{E},\overrightarrow{p_1})=-\frac{\pi\varepsilon_0 p^2}{2h^3}\left(1+\cos^2 \theta\right).
$$
The work to be done to remove the dipole to infinity is half of that:
$$
  A=-\frac{W}{2}=\frac{\pi\varepsilon_0 p^2}{4h^3}\left(1+\cos^2 \theta\right).
$$

5. Consider $d_1,d_2\ll L_1,L_2$ so that the plates form a type of a capacitor. Let $Ox$ axis be along the $L_2$-side, with $x=0$ at the left pair of sides (see picture). Then the electric field inside is
$$
  E(x)=\frac{V}{d_1 \left(1-\dfrac{x}{L_2}\right)+d_2 \dfrac{x}{L_2}}.
$$
The energy density is $w=\varepsilon_0 E^2/2$, so the energy is
$$
  W=\frac{\varepsilon_0 L_1}{2}\int\limits_0^{L_2} \left(d_1 \left(1-\frac{x}{L_2}\right)+d_2 \frac{x}{L_2}\right)E^2(x)dx=\frac{\varepsilon_0 V^2 L_1}{2}\int\limits_0^{L_2} \frac{dx}{d_1+\left(d_2-d_1\right)\dfrac{x}{L_2}}=\frac{\varepsilon_0 V^2 L_1 L_2}{2\left(d_2-d_1\right)}\ln\frac{d_2}{d_1}.
$$
By definition of capacitance the energy is $W=CV^2/2$, then the capacitance is
$$
  C=\frac{\varepsilon_0 L_1 L_2}{d_2-d_1}\ln\frac{d_2}{d_1}.
$$
In the limit of $d_1\to d_2=d$ the expression becomes
$$
  C=\frac{\varepsilon_0 L_1 L_2}{d},
$$
as for an ordinary capacitor.

\end{document}