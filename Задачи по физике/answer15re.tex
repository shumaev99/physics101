\documentclass[14pt,a4paper,pdflatex]{disser}
%\usepackage[russian]{babel}
%\usepackage[utf-8]{inputenc}
%\usepackage{amsmath,amssymb}
%\usepackage{longtable}
\usepackage{parskip}
\usepackage{caption}
\usepackage{textcomp}
\usepackage{gensymb}
\usepackage[dvips]{graphicx}
%\usepackage{wrapfig}
%\usepackage{amssymb}*
%\usepackage{color}
%\usepackage{ulem}
\usepackage{setspace}
\usepackage{hyperref}
\usepackage{pdfpages}

\oddsidemargin=0 cm
\evensidemargin=0 cm
\textwidth=170 mm
\textheight=260 mm
\topmargin=0 cm
\voffset= -2cm
\pagenumbering{true}
%\newlength{\varheight}
%\setlength{\varheight}{3.1cm}
\setlength{\parindent}{0cm}
%\newcommand{\taskname}[name]{\begin{center} \bf{\Large{name}} \end{center}}
\spacing{1.1}
\parskip=2mm
%\captionsetup[figure]{labelformat=empty}
\clubpenalty=10000
\widowpenalty=10000

\begin{document}

\includepdf{answer15_question1.pdf}
\pagebreak

\includepdf{answer15_question2.pdf}
\pagebreak

1. Sorry, there was a misspelling.. The correct answer is
$$
  \Omega=\sqrt{\frac{g^2}{\omega^2 l^2}-\omega^2},
$$
with a correct dimension.

2. Yes, that type of oscillations can exist and its frequency is $\omega_0 \sqrt{3}$. I don't see any obstacles to its existence.

\end{document} 