\documentclass[12pt,a4paper,pdflatex]{disser}
\usepackage[russian]{babel}
%\usepackage[utf-8]{inputenc}
%\usepackage{amsmath,amssymb}
%\usepackage{longtable}
\usepackage{parskip}
\usepackage{caption}
\usepackage{textcomp}
\usepackage{gensymb}
\usepackage[dvips]{graphicx}
%\usepackage{wrapfig}
%\usepackage{amssymb}*
%\usepackage{color}
%\usepackage{ulem}
\usepackage{setspace}
\usepackage{hyperref}

\oddsidemargin=0 cm
\evensidemargin=0 cm
\textwidth=170 mm
\textheight=260 mm
\topmargin=0 cm
\voffset= -2cm
\pagenumbering{true}
%\newlength{\varheight}
%\setlength{\varheight}{3.1cm}
\setlength{\parindent}{0cm}
%\newcommand{\taskname}[name]{\begin{center} \bf{\Large{name}} \end{center}}
\spacing{1.1}
\parskip=2mm
%\captionsetup[figure]{labelformat=empty}
\clubpenalty=10000
\widowpenalty=10000

\begin{document}

1. The distribution of the surface charge density $\sigma(\varphi)$ may be found from an integral-functional equation
$$
  \int\limits_0^\pi \sigma(\psi)\frac{\sin\psi d\psi}{\cos\varphi-\cos\psi}=0,
$$
valid for all $\varphi\in (0,\pi)$ (here $\varphi$ is the angle between the radius-vector of the point and the plane which bounds the potentials $V$ and $-V$). Then, for a given point $\overrightarrow{r}=(\rho,\varphi)$ the potential
$$
  \phi(\rho,\varphi)=C \int\limits_0^\pi \sigma(\psi)\left(\frac{1}{r_-}-\frac{1}{r_+}\right)d\psi,
$$
where
$$
  r_\pm=\sqrt{(r\cos\psi-\rho\cos\varphi)^2+(r\sin\psi\pm\rho\sin\varphi)^2
$$
and $C$ is a constant of dimension m$^3$/F and is dependent on $V$.

2. Obviously, due to the symmetry of the problem, the equipotential planes all will coincide on the axis of the system. Let's find the resistance $dR$ of a radial slice with an angle $d\alpha$ between the two bounding equipotential planes, if the thickness of the slit is $h$. Let the potential difference across the slice be $d\varphi$. Then the distribution of the electric field is
$$
  E(r)=\frac{d\varphi}{rd\alpha}.
$$
Then the distribution of the current density is
$$
  j(r)=\frac{E(r)}{\rho}=\frac{d\varphi}{\rho rd\alpha}.
$$
That gives the full current:
$$
  I=h\int\limits_{r_1}^{r_2} j(r)dr=\frac{hd\varphi}{\rho d\alpha}\ln\frac{r_2}{r_1},
$$
where $r_1=a$ and $r_2=2a$. The by definition the resistance is
$$
  dR=\frac{d\varphi}{I}=\frac{\rho}{h\ln2}d\alpha.
$$
Last, the full resistance is
$$
  R=2\pi\frac{dR}{d\alpha}=\frac{2\pi\rho}{h\ln2}.
$$

3. As the emitted frequency is equal to the cyclotron frequency, then
$$
  \omega=\frac{eB}{m},
$$
which implies
$$
  B=\frac{m\omega}{e}=1.6 \text{ mT}.
$$
As we see the radiation, the electrons must have a component of the velocity directed towards us (or in the opposite direction). Then a component of the magnetic field perpendicular to our view must exist. Nothing can be said about the radial (parallel to our view) component.

4. Consider the distribution of the current axial-symmetric. Due to the Maxwell equations, the field is maximal on the surface and equal to
$$
  B=\frac{\mu_0 I}{\pi d}.
$$
If the field is bounded by $B_\text{max}=0.1$ T, then the current is bounded by
$$
  I_\text{max}=\frac{\pi B_\text{max}d}{\mu_0}=750 \text{ A}.
$$ 

5. The acceleration due to the electric field is
$$
  a_E=\frac{eE}{m}=9.58\cdot 10^{10} \text{ m/s}^2,
$$
the acceleration due to the magnetic field is
$$
  a_B=\frac{eBv}{m}=1.20\cdot 10^{11} \text{ m/s}^2.
$$
As it can be seen on the picture, the electric acceleration is directed upwards, whereas the magnetic acceleration is directed downwards. Then the resulting acceleration is
$$
  a=a_B-a_E=2.40\cdot 10^{10} \text{ m/s}^2
$$
and is directed downwards. If the velocity of the antiproton were reversed, then the resulting acceleration would be
$$
  a=a_E+a_B=2.16\cdot 10^{11} \text{ m/s}^2,
$$
directed upwards.

6. Let's first solve a simpler problem. Consider a circular loop with current $i$ and radius $r$. Then the magnetic field in its center is found by Biot-Savart law:
$$
  B=\frac{\mu_0 i}{2r}.
$$
Let's return to our problem. The disk may be split into thin rings with thickness $dr$, then on the radius $r$ it provides a current
$$
  di=\frac{Q}{\pi R^2}\omega rdr,
$$
which creates a field
$$
  dB=\frac{\mu_0 di}{2r}=\frac{\mu_0 \omega Q}{2\pi R^2}dr.
$$
The field created by the whole disk may be obtained by integration of the previous expression:
$$
  B=\frac{\mu_0 \omega Q}{2\pi R}.
$$

\end{document} 