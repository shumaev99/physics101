\documentclass[14pt,a4paper,pdflatex]{disser}
\usepackage[russian]{babel}
%\usepackage[utf-8]{inputenc}
%\usepackage{amsmath,amssymb}
%\usepackage{longtable}
\usepackage{parskip}
\usepackage{caption}
\usepackage{textcomp}
\usepackage{gensymb}
\usepackage[dvips]{graphicx}
%\usepackage{wrapfig}
%\usepackage{amssymb}
%\usepackage{color}
%\usepackage{ulem}
\usepackage{setspace}
\usepackage{hyperref}

\oddsidemargin=0 cm
\evensidemargin=0 cm
\textwidth=170 mm
\textheight=260 mm
\topmargin=0 cm
\voffset= -2cm
\pagenumbering{false}
%\newlength{\varheight}
%\setlength{\varheight}{3.1cm}
\setlength{\parindent}{0cm}
%\newcommand{\taskname}[name]{\begin{center} \bf{\Large{name}} \end{center}}
\spacing{1.1}
\parskip=2mm
%\captionsetup[figure]{labelformat=empty}
\clubpenalty=10000
\widowpenalty=10000

\begin{document}

1. The energy of Earth --- Moon system decreases, same as the energy of the Mars with his asteroids. But the Mars doesn't have a hydrosphere and possesses a typical behavior --- as energy dissipates, the satellite's radius decreases.

Earth, in opposite, has a mobile hydrosphere. It creates a dissipation of Earth's rotational energy, but also it creates a momentum which speeds the Moon up! The situation is discussed in problem 3 of the IPhO 2009.

This is the situation of a non-ideal energy transfer machine (with Earth's hydro-\linebreak sphere as a working body) which transfers Earth's rotational energy to the Moon with heat losses due to oceanic friction. The ocean plays the role of an inductance (inertia) while the friction possesses itself as a resistance.

2. Generally, solar eclipses are almost as probable as lunar ones. The main fact is that the lunar eclipse is in average observed by \textit{\textbf{the half of Earth's population}}, while the effective radius of the solar eclipse is less than 10 km. So, for a particular location, solar eclipses are much more rare than lunar ones.

3. Generally, the answer depends on the way how the ground observer performs the measurement. Let's consider the simplest situation.

Due to relativistic contraction the length of the train is reduced by a factor of $l_0/l=\gamma=\left(1-\beta^2\right)^{-1/2}=2.29$. Then the rest length is $l_0=\gamma l=45.9$ m.

4. Due to relativistic time dilation, if the time interval in the rest frame is $\tau_0$, then the steady observer watches the moving clock and recognises the time inteval by the moving clock $\tau=\tau_0/\gamma=\tau_0\sqrt{1-\beta^2}$. If the moving clock's rate is the half of the rest clock's rate then the factor is $\gamma=2$ and the moving clock's speed is $\beta=\sqrt{1-1/\gamma^2}=0.87$.

\end{document} 