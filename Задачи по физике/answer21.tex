\documentclass[12pt,a4paper,pdflatex]{disser}
\usepackage[russian]{babel}
%\usepackage[utf-8]{inputenc}
%\usepackage{amsmath,amssymb}
%\usepackage{longtable}
\usepackage{parskip}
\usepackage{caption}
\usepackage{textcomp}
\usepackage{gensymb}
\usepackage[dvips]{graphicx}
%\usepackage{wrapfig}
%\usepackage{amssymb}*
%\usepackage{color}
%\usepackage{ulem}
\usepackage{setspace}
\usepackage{hyperref}

\oddsidemargin=-0.5 cm
\evensidemargin=-0.5 cm
\textwidth=170 mm
\textheight=260 mm
\topmargin=0 cm
\voffset= -2cm
\pagenumbering{true}
%\newlength{\varheight}
%\setlength{\varheight}{3.1cm}
\setlength{\parindent}{0cm}
%\newcommand{\taskname}[name]{\begin{center} \bf{\Large{name}} \end{center}}
\spacing{1.1}
\parskip=2mm
\captionsetup[figure]{labelformat=empty}
\clubpenalty=10000
\widowpenalty=10000

\begin{document}

2. The critical current is
$$
  I_c=\frac{\pi Bd}{\mu_0}=500 \text{ A},
$$
hence the critical current density is
$$
  j_c=\frac{4B}{\mu_0 d}=6.37\cdot 10^8 \text{ A/m}^2.
$$
For the tinobium wire the parameters are $I_c=61.3$ kA and $j_c=7.80\cdot 10^{10}$ A/m$^2$.

3. The conducting plane creates three image charges, which attract the initial charge with a force
$$
  F=\frac{q^2}{4\pi\varepsilon_0}\left(\frac{1}{4p^2}+\frac{4p^3 a^3}{\left(p^4-a^4\right)^2}\right).
$$

4. The seismic mass may be defined from the density of steel $\rho=7800$ kg/m$^3$ and is equal to $m=6.13$ g. The spring constant of the piezodisk is $k=E\pi r^2/d=1.39\cdot 10^9$ N/m, its capacitance is $C=\varepsilon_r \varepsilon_0 \pi r^2/d=7.82\cdot 10^{-11}$ F. So, the displacement of the disk is $x=m(g+a)/k$, and the voltage on it is
$$
  V=\frac{d_{yy}x}{C}=\frac{m(g+a) d_{yy}}{Ck}.
$$
So, the vertical acceleration of the seismic mass is
$$
  a=\frac{kCV}{md_{yy}}-g.
$$

\end{document} 