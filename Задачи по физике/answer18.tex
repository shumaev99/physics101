\documentclass[12pt,a4paper,pdflatex]{disser}
\usepackage[russian]{babel}
%\usepackage[utf-8]{inputenc}
%\usepackage{amsmath,amssymb}
%\usepackage{longtable}
\usepackage{parskip}
\usepackage{caption}
\usepackage{textcomp}
\usepackage{gensymb}
\usepackage[dvips]{graphicx}
%\usepackage{wrapfig}
%\usepackage{amssymb}*
%\usepackage{color}
%\usepackage{ulem}
\usepackage{setspace}
\usepackage{hyperref}

\oddsidemargin=0 cm
\evensidemargin=0 cm
\textwidth=170 mm
\textheight=260 mm
\topmargin=0 cm
\voffset= -2cm
\pagenumbering{true}
%\newlength{\varheight}
%\setlength{\varheight}{3.1cm}
\setlength{\parindent}{0cm}
%\newcommand{\taskname}[name]{\begin{center} \bf{\Large{name}} \end{center}}
\spacing{1.1}
\parskip=2mm
%\captionsetup[figure]{labelformat=empty}
\clubpenalty=10000
\widowpenalty=10000

\begin{document}

1. The magnetic field inside the solenoid is determined by the formula
$$
  B=\frac{\mu_0 I}{d},
$$
where $d=2$ mm is the distance between neighboring wires. This implies
$$
  I=\frac{Bd}{\mu_0}=2.39 \text{ kA}.
$$

2. Considering the change of the melting temperature small, we get
$$
  \Delta p\approx \frac{\lambda\rho_w \rho_i \Delta T}{\rho_w-\rho_i}=3.7 \text{ GPa},
$$
where $\lambda=336$ kJ/kg is the specific heat of melting for ice.

\end{document}