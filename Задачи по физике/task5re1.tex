\documentclass[14pt]{article}

\usepackage[utf8]{inputenc}
\usepackage{amsmath,amssymb}
\usepackage{parskip}
\usepackage{caption}
\usepackage{textcomp}
\usepackage{gensymb}
\usepackage[dvips]{graphicx}
\usepackage{wrapfig}
\usepackage{color}
\usepackage{setspace}
\usepackage{hyperref}
\usepackage{epstopdf}

\oddsidemargin=0 cm
\evensidemargin=0 cm
\textwidth=170 mm
\textheight=230 mm
\topmargin=0 cm
\voffset= -2cm
\pagenumbering{false}
\newlength{\varheight}
\setlength{\varheight}{3.1cm}
\setlength{\parindent}{0cm}
\spacing{1.1}
\parskip=2mm
\clubpenalty=10000
\widowpenalty=10000

\begin{document}

Maglev trains based on the so-called \textit{Inductrack} technology move similar to the magnet above a superconductive plane. For a Maglev train the drag force is
$$
  F_d=F_1 \frac{v_0 v}{v_0^2+v^2},
$$
and the lift force is
$$
  F_l=F_2\frac{v^2}{v_0^2+v^2},
$$
where $v_0$ is the characteristic velocity. The drag force reaches its maximum value $F_d (\text{max})=F_1/2$ at $v=v_0$ and tends to zero as $1/v^2$ at $v\gg v_0$. The lift force tends to $F_2$ as $v$ approaches infinity.

The qualitative description of the Maglev drag force is like this. As the dissipation is due to resistivity, it is analogous to viscous friction and gives a linear $v$ term in the numerator. The denominator is actually the characteristic impedance squared $Z^2$. The resistivity gives a constant term, the inductance gives the term that is proportional to $v^2$, aka $(\omega L)^2$.

Reference: WoPhO problem ``Why Maglev Trains Levitate'' (\href{http://www.wopho.org/file/upload/Theoretical2problem.pdf}{2011.2}).

\end{document} 