\documentclass[12pt,a4paper,pdflatex]{disser}
\usepackage[russian]{babel}
%\usepackage[utf-8]{inputenc}
%\usepackage{amsmath,amssymb}
%\usepackage{longtable}
\usepackage{parskip}
\usepackage{caption}
\usepackage{textcomp}
\usepackage{gensymb}
\usepackage[dvips]{graphicx}
%\usepackage{wrapfig}
%\usepackage{amssymb}*
%\usepackage{color}
%\usepackage{ulem}
\usepackage{setspace}
\usepackage{hyperref}

\oddsidemargin=-0.5 cm
\evensidemargin=-0.5 cm
\textwidth=170 mm
\textheight=260 mm
\topmargin=0 cm
\voffset= -2cm
\pagenumbering{true}
%\newlength{\varheight}
%\setlength{\varheight}{3.1cm}
\setlength{\parindent}{0cm}
%\newcommand{\taskname}[name]{\begin{center} \bf{\Large{name}} \end{center}}
\spacing{1.1}
\parskip=2mm
\captionsetup[figure]{labelformat=empty}
\clubpenalty=10000
\widowpenalty=10000

\begin{document}

1. The velocity at the first cross section is
$$
  v_1=\frac{Q}{7.854\cdot 10^{-3} \text{ m}^2}.
$$
The velocity at the second cross section is
$$
  v_2=\frac{4Q}{\pi d^2}.
$$
Due to Bernoulli's law,
$$
  v_1^2-v_2^2=2g\Delta h=3.92 \text{ m}^2/\text{s}^2.
$$
By substituting the velocities, we get
$$
  Q^2 \left(10^4 \text{ m}^{-4}+\frac{1}{d^4}\right)=2.418 \text{ m}^2/\text{s}^2,
$$
which implies
$$
  Q=\sqrt{2.418 \text{ m}^2/\text{s}^2 \cdot\left(10^4 \text{ m}^{-4}+\frac{1}{d^4}\right)^{-1}}.
$$

2. The potential energy of a mass $m$ on such a planet on a height $h$ is
$$
  W(h)=mh\left(g_0-\frac{\alpha h}{2}\right).
$$
The the Bernoulli equation looks like
$$
  p+\frac{\rho v^2}{2}+\rho h\left(g_0-\frac{\alpha h}{2}\right).
$$

3. The air masses move circumferentially under the influence of the pressure gradient. If the velocity pattern is $v(r)$, then the Newton's law states 
$$
  \frac{\rho v^2}{r}=\frac{dp}{dr}.
$$
It's natural to assume that the Bernoulli's law works for all the range of $r$, although it's a controversial issue. If it's true then
$$
  p+\frac{\rho v^2}{2}=const.
$$
Differentiation yields
$$
  \frac{dp}{dr}+\rho v \frac{dv}{dr}=0.
$$
Comparing this equation with the Newton's, we get
$$
  \frac{dv}{v}=-\frac{dr}{r},
$$
which solution is
$$
  v(r)\propto r^{-1},
$$
or
$$
  v(r)=v_0 \frac{r_0}{r},
$$
where $v_0$ and $r_0$ are some constants. As it can be seen, this solution is absurd in the range of small $r$, that happens because Bernoulli's law works bad for big velocities (actually, it works bad everywhere, the truth is that it's \textit{never} true). Maybe, a better solution would be obtained from the Navier-Stokes equation.

\end{document} 