\documentclass[14pt]{article}

\usepackage[russian]{babel}
\usepackage{amsmath,amssymb}
\usepackage{parskip}
\usepackage{caption}
\usepackage{textcomp}
\usepackage{gensymb}
\usepackage[dvips]{graphicx}
\usepackage{wrapfig}
\usepackage{color}
\usepackage{setspace}
%\usepackage{hyperref}
\usepackage{epstopdf}

\oddsidemargin=0 cm
\evensidemargin=0 cm
\textwidth=170 mm
\textheight=230 mm
\topmargin=0 cm
\voffset= -2cm
\pagenumbering{false}
\newlength{\varheight}
\setlength{\varheight}{3.1cm}
\setlength{\parindent}{0cm}
\spacing{1.1}
\parskip=2mm
\clubpenalty=10000
\widowpenalty=10000

\begin{document}

0. The top card overhangs the second one by a maximum of a half of a card length, the second one a fourth over the third one, the third one the sixth over the fourth one and so on. So for $N+1$ cards the total overhang is
$$
  L=\frac{a}{2}\sum\limits_{i=1}^N \frac{1}{i},
$$
where $a=88$ mm is the card length.

1. The field in the dielectric is reduced by $\varepsilon/\varepsilon_0$ times. So, the field in the first dielectric is
$$
  E_1=\frac{q}{4\pi\varepsilon_1 r^2},
$$
and in the second one
$$
  E_2=\frac{q}{4\pi\varepsilon_2 r^2}.
$$
The potential is respective for these fields.

2. Same as previous problem, the fields are
$$
  E_i=\frac{q}{4\pi \varepsilon_i r^2},
$$
the potentials are
$$
  \varphi_i=\frac{q}{4\pi\varepsilon_i r}.
$$

3. By introducing the new variable $t=a\sqrt{x}$ this expression simplifies to
$$
  \int\limits_0^\infty \frac{tdt}{1+e^t}=\frac{\pi^2}{12}.
$$

4. By differentiation with respect to $x$ and $y$ we get
\begin{gather}
  (x+z^2)\frac{\partial z}{\partial x}+z=0,\nonumber\\
  (x+z^2)\frac{\partial^2 x}{\partial z^2}+2\frac{\partial z}{\partial x}+2z\left(\frac{\partial z}{\partial x}\right)^2=0,\nonumber\\
  (x+z^2)\frac{\partial z}{\partial y}=1,\nonumber\\
  (x+z^2)\frac{\partial^2 z}{\partial y^2}+2z\left(\frac{\partial z}{\partial y}\right)^2=0.\nonumber
\end{gather}
Eliminate the first derivatives:
\begin{gather}
  \frac{\partial^2 z}{\partial x^2}=\frac{2xz}{\left(x+z^2\right)^3},\nonumber\\
  \frac{\partial^2 z}{\partial y^2}=-\frac{2z}{\left(x+z^2\right)^3}.\nonumber
\end{gather}
These satisfy the required equation.

5. The equation for $\beta_n$ is
$$
  \beta_{k+1}=\frac{\beta_k+\beta}{1+\beta_k \beta}
$$
with $\beta_0=0$. The solution is
$$
  \beta_n=\frac{1-a^n}{1+a^n},
$$
where
$$
  a=\frac{1-\beta}{1+\beta}.
$$
The infinity asymptotic is
$$
  \beta_n\approx 1-2a^n.
$$
There is another approach to the solution. The velocity summation law of special relativity states that the velocities $\beta_i$ are not additive but the quantities $\tanh^{-1}\beta_i$ are. So,
$$
  \tanh^{-1}\beta_{k+1}=\tanh^{-1}\beta_k+\tanh^{-1}\beta.
$$
This easily yields
$$
  \beta_n=\tanh\left(n\tanh^{-1}\beta\right).
$$

\end{document} 