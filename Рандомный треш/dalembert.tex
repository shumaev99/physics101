\documentclass[14pt]{article}

\usepackage[ukrainian]{babel}
\usepackage[utf8]{inputenc}
\usepackage{amsmath,amssymb}
\usepackage{parskip}
\usepackage{caption}
\usepackage{textcomp}
\usepackage{gensymb}
\usepackage[dvips]{graphicx}
\usepackage{wrapfig}
\usepackage{color}
\usepackage{setspace}
%\usepackage{hyperref}
\usepackage{epstopdf}

\oddsidemargin=0 cm
\evensidemargin=0 cm
\textwidth=170 mm
\textheight=230 mm
\topmargin=0 cm
\voffset= -2cm
\pagenumbering{false}
\newlength{\varheight}
\setlength{\varheight}{3.1cm}
\setlength{\parindent}{0cm}
\spacing{1.1}
\parskip=2mm
\clubpenalty=10000
\widowpenalty=10000

\begin{document}

Маємо неоднорідне рівняння Даламбера на функцію $u(x,y,t), (x,y)\in \mathbb{R}^2, t>0$, з початковими умовами:
\begin{gather*}
  u_{tt}=c^2(u_{xx}+u_{yy})+f(x,y,t),\\
  u(x,y,0)=\varphi(x,y),\\
  u_t(x,y,0)=\psi(x,y).
\end{gather*}
Розв'язок цього рівняння відомий, але я спробував його отримати. Перейдемо до фур'є-зображення $U(\omega_x,\omega_y)$:
\begin{gather*}
  U_{tt}=-\omega^2 c^2 U+F(\vec{\omega},t),\\
  U(0)=\Phi(\vec{\omega}),\\
  U_t(0)=\Psi(\vec{\omega}).
\end{gather*}
Тепер зробимо перетворення Лапласа по $t$:
$$
  s^2 \mathcal{U}-s\Phi-\Psi=-\omega^2 c^2 \mathcal{U}+\mathcal{F}(\vec{\omega},s),
$$
звідки
$$
  \mathcal{U}=\frac{\mathcal{F}+s\Phi+\Psi}{s^2+\omega^2 c^2}.
$$
Зробимо обернене перетворення Лапласа:
$$
  U=\Phi(\vec{\omega})\frac{\sin\omega ct}{\omega c}+\Psi(\vec{\omega}) \cos \omega ct +\int\limits_0^t F(\vec{\omega},t-\tau)\frac{\sin \omega c\tau}{\omega c}d\tau.
$$
Тепер треба робити обернене перетворення Фур'є. Звичайно, для заданих початкових умов це зробити нескладно, але як отримати загальну відповідь, при тому що $\cos \omega ct$ не перетворюється?

\end{document} 